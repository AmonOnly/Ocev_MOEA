% ============================================================================
% RESUMO
% ============================================================================

\section*{Resumo}
\addcontentsline{toc}{section}{Resumo}

Este relatório apresenta uma análise experimental detalhada do algoritmo NSGA-II (\textit{Non-dominated Sorting Genetic Algorithm II}) aplicado aos problemas de benchmark ZDT1 e ZDT3. O estudo compara três variantes do NSGA-II com o algoritmo \textit{Random Search}, utilizando métricas quantitativas de performance (\textit{Hypervolume} e \textit{Spacing}) e análise de convergência ao longo de 250 gerações.

\subsection*{Objetivos}

\begin{itemize}
    \item Avaliar a eficácia do NSGA-II em problemas multiobjetivo com diferentes características topológicas (ZDT1 contínuo vs. ZDT3 descontínuo)
    \item Analisar o impacto do mecanismo de \textit{crowding distance} na diversidade das soluções
    \item Investigar o efeito da normalização fixa de limites (\textit{fixed bounds}) no desempenho do algoritmo
    \item Comparar quantitativamente algoritmos evolutivos com busca aleatória
    \item Estudar a dinâmica de convergência através de rastreamento em tempo real
\end{itemize}

\subsection*{Principais Resultados}

Os experimentos revelaram que:

\begin{itemize}
    \item \textbf{Superioridade do NSGA-II}: O algoritmo atingiu valores de \hlv{} aproximadamente 100 vezes superiores ao \textit{Random Search}, com $HV_{ZDT1} = 0.964$ e $HV_{ZDT3} = 1.369$
    
    \item \textbf{Convergência rápida}: O NSGA-II alcançou 90\% do \hlv{} final em apenas 45 gerações (18\% do total), demonstrando eficiência computacional
    
    \item \textbf{Importância do \textit{crowding distance}}: A remoção deste mecanismo resultou em queda de 30\% no \hlv{} e aumento de 87.5\% no \spac{}, evidenciando perda significativa de diversidade
    
    \item \textbf{Impacto mínimo da normalização fixa}: A diferença de performance entre NSGA-II padrão e com \textit{fixed bounds} foi inferior a 1\%, indicando robustez do algoritmo
    
    \item \textbf{Adaptabilidade a diferentes topologias}: O NSGA-II demonstrou desempenho consistente tanto no problema contínuo (ZDT1) quanto no descontínuo (ZDT3)
\end{itemize}

\subsection*{Contribuições}

Este estudo contribui com:

\begin{enumerate}
    \item Análise quantitativa detalhada com 10 execuções independentes por configuração
    \item Sistema de rastreamento de convergência em tempo real, coletando métricas em todas as 250 gerações
    \item Visualizações abrangentes incluindo fronteiras de Pareto, análises estatísticas e curvas de convergência
    \item Estudo comparativo sistemático de variantes do NSGA-II
    \item Validação experimental dos mecanismos fundamentais do NSGA-II
\end{enumerate}

\textbf{Palavras-chave}: Otimização multiobjetivo, NSGA-II, ZDT benchmark, \textit{Hypervolume}, \textit{Spacing}, \textit{crowding distance}, análise de convergência.
