% ============================================================================
% INTRODUÇÃO
% ============================================================================

\section{Introdução}

\subsection{Contextualização}

Problemas de otimização multiobjetivo (MOO - \textit{Multi-Objective Optimization}) são ubíquos em aplicações de engenharia, ciências e gestão, caracterizando-se pela necessidade de otimizar simultaneamente múltiplos objetivos conflitantes. Diferentemente da otimização mono-objetivo, onde existe uma única solução ótima, em MOO busca-se um conjunto de soluções de compromisso conhecidas como fronteira de Pareto.

A natureza NP-difícil da maioria dos problemas MOO reais torna inviável a aplicação de métodos exatos, motivando o desenvolvimento de meta-heurísticas baseadas em população. Entre estas, os Algoritmos Evolutivos Multiobjetivo (MOEAs - \textit{Multi-Objective Evolutionary Algorithms}) destacam-se pela capacidade de aproximar a fronteira de Pareto completa em uma única execução.

\subsection{O Algoritmo NSGA-II}

O NSGA-II (\textit{Non-dominated Sorting Genetic Algorithm II}), proposto por Deb et al. (2002), representa um marco na área de otimização evolutiva multiobjetivo. O algoritmo introduziu três mecanismos fundamentais:

\begin{enumerate}
    \item \textbf{Ordenação não-dominada rápida (\textit{fast non-dominated sorting})}: Classifica a população em fronts de não-dominância com complexidade $O(MN^2)$, onde $M$ é o número de objetivos e $N$ o tamanho da população
    
    \item \textbf{Distância de aglomeração (\textit{crowding distance})}: Estima a densidade de soluções ao redor de cada indivíduo, preservando diversidade sem necessidade de parâmetros de nicho
    
    \item \textbf{Elitismo}: Combina população parental e filha, garantindo preservação das melhores soluções ao longo das gerações
\end{enumerate}

Estas inovações resultaram em desempenho superior aos algoritmos predecessores (MOGA, NPGA, NSGA), tornando o NSGA-II referência na literatura e aplicações práticas.

\subsection{Problemas de Benchmark ZDT}

A suíte ZDT (Zitzler, Deb \& Thiele, 2000) constitui um conjunto padronizado de problemas de teste para MOEAs, permitindo avaliação sistemática e comparação entre algoritmos. Este estudo foca em dois problemas representativos:

\subsubsection{ZDT1 - Problema Contínuo}

Caracteriza-se por:
\begin{itemize}
    \item Fronteira de Pareto convexa e contínua
    \item 50 variáveis inteiras no intervalo $[0, 1000]$
    \item Objetivos: minimizar $f_1(x)$ e $f_2(x)$
    \item Topologia que favorece convergência uniforme
\end{itemize}

\subsubsection{ZDT3 - Problema Descontínuo}

Distingue-se por:
\begin{itemize}
    \item Fronteira de Pareto descontínua com 5 regiões separadas
    \item 50 variáveis reais no intervalo $[0, 1]$
    \item Objetivos: minimizar $f_1(x)$ e $f_2(x)$
    \item Desafio adicional de manter diversidade em múltiplas regiões
\end{itemize}

\subsection{Motivação do Estudo}

Apesar da ampla adoção do NSGA-II, questões fundamentais sobre seus mecanismos permanecem relevantes:

\begin{itemize}
    \item Qual o impacto quantitativo do \textit{crowding distance} na qualidade e diversidade das soluções?
    \item Como diferentes esquemas de normalização afetam o desempenho em problemas com características distintas?
    \item Qual a dinâmica de convergência do algoritmo ao longo das gerações?
    \item Quão superior é uma abordagem evolutiva comparada à busca aleatória?
\end{itemize}

Este estudo visa responder estas questões através de experimentação rigorosa e análise quantitativa.

\subsection{Objetivos Específicos}

\begin{enumerate}
    \item Implementar e validar o NSGA-II conforme especificação original
    \item Avaliar três variantes: padrão, com normalização fixa, e sem \textit{crowding distance}
    \item Comparar com baseline de \textit{Random Search}
    \item Aplicar nos problemas ZDT1 e ZDT3
    \item Medir performance através de \hlv{} e \spac{}
    \item Analisar convergência com rastreamento em tempo real (250 gerações)
    \item Realizar análise estatística com 10 execuções independentes
    \item Produzir visualizações abrangentes para interpretação dos resultados
\end{enumerate}

\subsection{Estrutura do Relatório}

O restante deste documento está organizado como segue:

\begin{itemize}
    \item \textbf{Seção 2}: Descreve a metodologia experimental detalhada
    \item \textbf{Seção 3}: Apresenta a configuração experimental e parâmetros
    \item \textbf{Seção 4}: Reporta resultados quantitativos e qualitativos
    \item \textbf{Seção 5}: Analisa a dinâmica de convergência
    \item \textbf{Seção 6}: Discute implicações e limitações
    \item \textbf{Seção 7}: Sintetiza conclusões e trabalhos futuros
\end{itemize}
