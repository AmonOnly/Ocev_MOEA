\section{Configuração Experimental}

\subsection{Parâmetros do NSGA-II}
A Tabela \ref{tab:params} resume os parâmetros utilizados consistentemente em todos os experimentos:

\begin{table}[h]
\centering
\begin{tabular}{ll}
\toprule
\textbf{Parâmetro} & \textbf{Valor} \\
\midrule
População & 100 indivíduos \\
Gerações & 250 \\
Probabilidade de crossover & 0.9 \\
Probabilidade de mutação & $1/N$ (adaptativa) \\
Operador de crossover & SBX ($\eta_c = 15$) \\
Operador de mutação & Polynomial ($\eta_m = 20$) \\
Seleção & Torneio binário \\
\bottomrule
\end{tabular}
\caption{Parâmetros do NSGA-II utilizados nos experimentos.}
\label{tab:params}
\end{table}

\subsection{Configurações de Dimensionalidade}
Testamos três níveis de dimensionalidade:
\begin{itemize}
  \item \textbf{N = 50}: Baseline (configuração padrão da literatura para ZDT).
  \item \textbf{N = 100}: Dimensionalidade intermediária (2× baseline).
  \item \textbf{N = 200}: Alta dimensionalidade (4× baseline).
\end{itemize}

\textbf{Nota sobre escalabilidade de parâmetros}: Mantivemos população e gerações fixas intencionalmente para isolar o efeito puro do aumento de N. A literatura sugere escalar população como Pop $\propto \sqrt{N}$ para compensar a maldição da dimensionalidade, o que será discutido nas recomendações (Seção 5).

\subsection{Infraestrutura Computacional}
\begin{itemize}
  \item \textbf{Linguagem}: Python 3.10+
  \item \textbf{Bibliotecas}: DEAP 1.4.1 (implementação do NSGA-II), NumPy 1.24+, Matplotlib 3.7+
  \item \textbf{Hardware}: CPU Intel Core i7, 16GB RAM
  \item \textbf{Tempo de execução}: ~30-40 minutos para completar os 60 experimentos (2 problemas × 3 valores de N × 10 runs)
\end{itemize}

\subsection{Organização dos Dados}
Os resultados são armazenados em \texttt{results\_nvar\_comparison/}:
\begin{itemize}
  \item 6 arquivos de resultados (ZDT1/ZDT3 × N50/N100/N200)
  \item Cada arquivo contém: HV médio, desvio-padrão, Spacing médio, desvio-padrão
  \item Relatório consolidado (\texttt{COMPARISON\_REPORT.txt}) com análise estatística
\end{itemize}

As visualizações são geradas em \texttt{plots/} no formato PDF (vetorial) para inclusão no relatório.