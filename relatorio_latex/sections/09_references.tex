% ============================================================================
% REFERÊNCIAS
% ============================================================================

\section{Referências}

\begin{enumerate}
    \item \textbf{Deb, K., Pratap, A., Agarwal, S., \& Meyarivan, T.} (2002). \textit{A fast and elitist multiobjective genetic algorithm: NSGA-II}. IEEE Transactions on Evolutionary Computation, 6(2), 182-197.
    
    \item \textbf{Zitzler, E., Deb, K., \& Thiele, L.} (2000). \textit{Comparison of multiobjective evolutionary algorithms: Empirical results}. Evolutionary Computation, 8(2), 173-195.
    
    \item \textbf{Deb, K.} (2001). \textit{Multi-Objective Optimization using Evolutionary Algorithms}. Wiley-Interscience Series in Systems and Optimization. John Wiley \& Sons.
    
    \item \textbf{Coello Coello, C. A., Lamont, G. B., \& Van Veldhuizen, D. A.} (2007). \textit{Evolutionary Algorithms for Solving Multi-Objective Problems} (2nd ed.). Springer.
    
    \item \textbf{Zitzler, E., \& Thiele, L.} (1999). \textit{Multiobjective evolutionary algorithms: A comparative case study and the strength Pareto approach}. IEEE Transactions on Evolutionary Computation, 3(4), 257-271.
    
    \item \textbf{While, L., Hingston, P., Barone, L., \& Huband, S.} (2006). \textit{A faster algorithm for calculating hypervolume}. IEEE Transactions on Evolutionary Computation, 10(1), 29-38.
    
    \item \textbf{Schott, J. R.} (1995). \textit{Fault Tolerant Design Using Single and Multicriteria Genetic Algorithm Optimization}. Master's Thesis, Massachusetts Institute of Technology, Department of Aeronautics and Astronautics.
    
    \item \textbf{Ishibuchi, H., Tsukamoto, N., \& Nojima, Y.} (2008). \textit{Evolutionary many-objective optimization: A short review}. In IEEE Congress on Evolutionary Computation (CEC), 2419-2426.
    
    \item \textbf{Deb, K., \& Jain, H.} (2014). \textit{An evolutionary many-objective optimization algorithm using reference-point-based nondominated sorting approach, Part I: Solving problems with box constraints}. IEEE Transactions on Evolutionary Computation, 18(4), 577-601.
    
    \item \textbf{Beume, N., Naujoks, B., \& Emmerich, M.} (2007). \textit{SMS-EMOA: Multiobjective selection based on dominated hypervolume}. European Journal of Operational Research, 181(3), 1653-1669.
    
    \item \textbf{Zhang, Q., \& Li, H.} (2007). \textit{MOEA/D: A multiobjective evolutionary algorithm based on decomposition}. IEEE Transactions on Evolutionary Computation, 11(6), 712-731.
    
    \item \textbf{Fortin, F. A., De Rainville, F. M., Gardner, M. A., Parizeau, M., \& Gagné, C.} (2012). \textit{DEAP: Evolutionary algorithms made easy}. Journal of Machine Learning Research, 13, 2171-2175.
    
    \item \textbf{Blank, J., \& Deb, K.} (2020). \textit{pymoo: Multi-Objective Optimization in Python}. IEEE Access, 8, 89497-89509.
    
    \item \textbf{Huband, S., Hingston, P., Barone, L., \& While, L.} (2006). \textit{A review of multiobjective test problems and a scalable test problem toolkit}. IEEE Transactions on Evolutionary Computation, 10(5), 477-506.
    
    \item \textbf{Li, M., \& Yao, X.} (2019). \textit{Quality evaluation of solution sets in multiobjective optimisation: A survey}. ACM Computing Surveys, 52(2), 1-38.
    
    \item \textbf{Knowles, J. D., Thiele, L., \& Zitzler, E.} (2006). \textit{A tutorial on the performance assessment of stochastic multiobjective optimizers}. TIK Report 214, Computer Engineering and Networks Laboratory (TIK), ETH Zurich.
    
    \item \textbf{Fonseca, C. M., \& Fleming, P. J.} (1995). \textit{An overview of evolutionary algorithms in multiobjective optimization}. Evolutionary Computation, 3(1), 1-16.
    
    \item \textbf{Miettinen, K.} (1999). \textit{Nonlinear Multiobjective Optimization}. International Series in Operations Research \& Management Science. Springer.
    
    \item \textbf{López-Ibáñez, M., Dubois-Lacoste, J., Cáceres, L. P., Birattari, M., \& Stützle, T.} (2016). \textit{The irace package: Iterated racing for automatic algorithm configuration}. Operations Research Perspectives, 3, 43-58.
    
    \item \textbf{Wolpert, D. H., \& Macready, W. G.} (1997). \textit{No free lunch theorems for optimization}. IEEE Transactions on Evolutionary Computation, 1(1), 67-82.
\end{enumerate}

\subsection{Recursos Online}

\begin{itemize}
    \item \textbf{DEAP Documentation}: \url{https://deap.readthedocs.io/}
    \item \textbf{pymoo Documentation}: \url{https://pymoo.org/}
    \item \textbf{ZDT Test Suite}: \url{https://en.wikipedia.org/wiki/Test_functions_for_optimization}
    \item \textbf{NSGA-II Original Paper}: \url{https://doi.org/10.1109/4235.996017}
    \item \textbf{Hypervolume Calculation}: \url{https://ls11-www.cs.tu-dortmund.de/rudolph/hypervolume/start}
\end{itemize}

\subsection{Ferramentas e Bibliotecas Utilizadas}

\begin{table}[H]
\centering
\caption{Principais ferramentas e bibliotecas utilizadas no estudo}
\begin{tabular}{@{}llp{6cm}@{}}
\toprule
\textbf{Ferramenta} & \textbf{Versão} & \textbf{Propósito} \\
\midrule
Python & 3.10+ & Linguagem de programação principal \\
DEAP & 1.4.1 & Framework de algoritmos evolutivos \\
pymoo & 0.6.1 & Cálculo de métricas (HV, Spacing) \\
NumPy & 1.24+ & Computação numérica e arrays \\
Matplotlib & 3.7+ & Geração de gráficos e visualizações \\
JSON & stdlib & Armazenamento de dados de convergência \\
\bottomrule
\end{tabular}
\end{table}

\subsection{Repositório de Código}

Todo o código-fonte, dados experimentais e scripts de análise estão disponíveis no repositório:

\begin{center}
\url{https://github.com/AmonOnly/Ocev_MOEA}
\end{center}

O repositório inclui:
\begin{itemize}
    \item Implementações dos 4 algoritmos testados
    \item Scripts de geração de dados de convergência
    \item Scripts de visualização (plotagem)
    \item Dados brutos (JSON) de 10 execuções finais + 3 execuções de convergência
    \item Todos os 8 gráficos em formato PNG (alta resolução) e PDF (vetorial)
    \item Documentação técnica (READMEs)
    \item Este relatório em LaTeX
\end{itemize}
