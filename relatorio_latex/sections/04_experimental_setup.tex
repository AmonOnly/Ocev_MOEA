% ============================================================================
% CONFIGURAÇÃO EXPERIMENTAL
% ============================================================================

\section{Configuração Experimental}

\subsection{Parâmetros dos Algoritmos}

A Tabela~\ref{tab:parameters} sumariza os parâmetros utilizados em todos os experimentos.

\begin{table}[H]
\centering
\caption{Parâmetros experimentais dos algoritmos evolutivos}
\label{tab:parameters}
\begin{tabular}{@{}lcc@{}}
\toprule
\textbf{Parâmetro} & \textbf{Símbolo} & \textbf{Valor} \\
\midrule
Tamanho da população & $N$ & 100 \\
Número de gerações & $G$ & 250 \\
Número de variáveis & $n$ & 50 \\
Probabilidade de cruzamento & $P_c$ & 0.9 \\
Probabilidade de mutação & $P_m$ & 0.1 \\
Parâmetro BLX-$\alpha$ & $\alpha$ & 0.5 \\
Tamanho do torneio & $k$ & 2 \\
\midrule
Total de avaliações & $N \times G$ & 25{,}000 \\
Execuções independentes & - & 10 \\
\bottomrule
\end{tabular}
\end{table}

\subsection{Especificação dos Problemas}

\subsubsection{ZDT1}

\textbf{Formulação matemática}:
\begin{align*}
\text{Minimizar } & f_1(x) = x_1 \\
                  & f_2(x) = g(x) \cdot h(f_1, g) \\
\text{onde } & g(x) = 1 + \frac{9}{n-1}\sum_{i=2}^{n} x_i \\
             & h(f_1, g) = 1 - \sqrt{\frac{f_1}{g}}
\end{align*}

\textbf{Características}:
\begin{itemize}
    \item Domínio: $x_i \in \{0, 1, 2, \ldots, 1000\}$ (variáveis inteiras)
    \item Normalização interna: $x_i / 1000$ para $x_i \in [0, 1]$
    \item Fronteira de Pareto: convexa, definida por $f_2 = 1 - \sqrt{f_1}$ com $f_1 \in [0, 1]$
    \item Dificuldade: convergência relativamente fácil, distribuição uniforme natural
\end{itemize}

\subsubsection{ZDT3}

\textbf{Formulação matemática}:
\begin{align*}
\text{Minimizar } & f_1(x) = x_1 \\
                  & f_2(x) = g(x) \cdot h(f_1, g) \\
\text{onde } & g(x) = 1 + \frac{9}{n-1}\sum_{i=2}^{n} x_i \\
             & h(f_1, g) = 1 - \sqrt{\frac{f_1}{g}} - \frac{f_1}{g}\sin(10\pi f_1)
\end{align*}

\textbf{Características}:
\begin{itemize}
    \item Domínio: $x_i \in [0, 1]$ (variáveis reais)
    \item Fronteira de Pareto: descontínua com 5 regiões separadas
    \item Regiões aproximadas: $f_1 \in [0, 0.083] \cup [0.182, 0.258] \cup [0.409, 0.454] \cup [0.618, 0.653] \cup [0.823, 0.852]$
    \item Dificuldade: manter diversidade em múltiplas regiões desconexas
\end{itemize}

\subsection{Configurações Avaliadas}

O experimento comparou 8 configurações distintas:

\begin{table}[H]
\centering
\caption{Matriz de configurações experimentais}
\label{tab:configurations}
\begin{tabular}{@{}lll@{}}
\toprule
\textbf{Algoritmo} & \textbf{ZDT1} & \textbf{ZDT3} \\
\midrule
NSGA-II padrão & \checkmark & \checkmark \\
NSGA-II com \textit{fixed bounds} & \checkmark & \checkmark \\
NSGA-II sem \textit{crowding distance} & \checkmark & \checkmark \\
\textit{Random Search} & \checkmark & \checkmark \\
\midrule
\textbf{Total de combinações} & \multicolumn{2}{c}{8} \\
\textbf{Execuções por combinação} & \multicolumn{2}{c}{10} \\
\textbf{Total de execuções} & \multicolumn{2}{c}{80} \\
\bottomrule
\end{tabular}
\end{table}

\subsection{Infraestrutura Computacional}

\textbf{Hardware}:
\begin{itemize}
    \item Processador: CPU x86\_64 (especificação variável)
    \item Memória: Suficiente para manipulação de populações
    \item Sistema operacional: Linux (Ubuntu/Debian)
\end{itemize}

\textbf{Software}:
\begin{itemize}
    \item Linguagem: Python 3.x
    \item Bibliotecas: NumPy (computação numérica), Matplotlib (visualização)
    \item Controle de versão: Git
    \item Ambiente: VS Code com extensões Python
\end{itemize}

\subsection{Protocolo Experimental}

\textbf{Fase 1 - Validação}:
\begin{enumerate}
    \item Implementação dos algoritmos
    \item Testes unitários dos componentes principais
    \item Verificação de conformidade com especificações
    \item Execuções preliminares para ajuste de parâmetros
\end{enumerate}

\textbf{Fase 2 - Coleta de Dados}:
\begin{enumerate}
    \item Execução de 10 runs por configuração (resultados finais)
    \item Execução de 3 runs com rastreamento de convergência
    \item Registro de fronteiras de Pareto finais
    \item Cálculo de métricas (HV e SP)
    \item Armazenamento estruturado de resultados
\end{enumerate}

\textbf{Fase 3 - Análise}:
\begin{enumerate}
    \item Computação de estatísticas descritivas
    \item Geração de visualizações
    \item Análise de significância de diferenças
    \item Identificação de padrões e tendências
\end{enumerate}

\subsection{Garantia de Qualidade}

Medidas adotadas para assegurar confiabilidade:

\begin{itemize}
    \item \textbf{Reprodutibilidade}: Sementes aleatórias documentadas
    \item \textbf{Validação cruzada}: Comparação com resultados da literatura
    \item \textbf{Múltiplas execuções}: 10 runs independentes
    \item \textbf{Versionamento}: Controle de versão de todo código
    \item \textbf{Documentação}: Comentários e documentação inline
    \item \textbf{Backup}: Preservação de implementações originais (.bak)
\end{itemize}

\subsection{Limitações Conhecidas}

\textbf{Escopo}:
\begin{itemize}
    \item Análise restrita a 2 problemas da suíte ZDT
    \item Problemas bi-objetivo apenas
    \item Comparação limitada a variantes do NSGA-II
    \item Ausência de testes estatísticos formais (e.g., Wilcoxon, Mann-Whitney)
\end{itemize}

\textbf{Computacional}:
\begin{itemize}
    \item Rastreamento de convergência reduzido para 3 runs (eficiência)
    \item Métricas calculadas em Python (não otimizado)
    \item Ausência de paralelização
\end{itemize}

Estas limitações são consideradas aceitáveis dado o escopo educacional e demonstrativo do estudo.
